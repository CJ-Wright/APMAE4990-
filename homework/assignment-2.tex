%%%%%%%%%%%%%%%%%%%%%%%%%%%%%%%%%%%%%%%%%
% Short Sectioned Assignment
% LaTeX Template
% Version 1.0 (5/5/12)
%
% This template has been downloaded from:
% http://www.LaTeXTemplates.com
%
% Original author:
% Frits Wenneker (http://www.howtotex.com)
%
% License:
% CC BY-NC-SA 3.0 (http://creativecommons.org/licenses/by-nc-sa/3.0/)
%
%%%%%%%%%%%%%%%%%%%%%%%%%%%%%%%%%%%%%%%%%

%----------------------------------------------------------------------------------------
%	PACKAGES AND OTHER DOCUMENT CONFIGURATIONS
%----------------------------------------------------------------------------------------

\documentclass[paper=a4, fontsize=11pt]{scrartcl} % A4 paper and 11pt font size

\usepackage[T1]{fontenc} % Use 8-bit encoding that has 256 glyphs
\usepackage{fourier} % Use the Adobe Utopia font for the document - comment this line to return to the LaTeX default
\usepackage[english]{babel} % English language/hyphenation
\usepackage{amsmath,amsfonts,amsthm} % Math packages
\usepackage{hyperref}
\usepackage{lipsum} % Used for inserting dummy 'Lorem ipsum' text into the template

\usepackage{sectsty} % Allows customizing section commands
\allsectionsfont{\centering \normalfont\scshape} % Make all sections centered, the default font and small caps

\usepackage{fancyhdr} % Custom headers and footers
\pagestyle{fancyplain} % Makes all pages in the document conform to the custom headers and footers
\fancyhead{} % No page header - if you want one, create it in the same way as the footers below
\fancyfoot[L]{} % Empty left footer
\fancyfoot[C]{} % Empty center footer
\fancyfoot[R]{\thepage} % Page numbering for right footer
\renewcommand{\headrulewidth}{0pt} % Remove header underlines
\renewcommand{\footrulewidth}{0pt} % Remove footer underlines
\setlength{\headheight}{13.6pt} % Customize the height of the header

\numberwithin{equation}{section} % Number equations within sections (i.e. 1.1, 1.2, 2.1, 2.2 instead of 1, 2, 3, 4)
\numberwithin{figure}{section} % Number figures within sections (i.e. 1.1, 1.2, 2.1, 2.2 instead of 1, 2, 3, 4)
\numberwithin{table}{section} % Number tables within sections (i.e. 1.1, 1.2, 2.1, 2.2 instead of 1, 2, 3, 4)

\setlength\parindent{0pt} % Removes all indentation from paragraphs - comment this line for an assignment with lots of text

%----------------------------------------------------------------------------------------
%	TITLE SECTION
%----------------------------------------------------------------------------------------

\newcommand{\horrule}[1]{\rule{\linewidth}{#1}} % Create horizontal rule command with 1 argument of height

\title{	
\normalfont \normalsize 
\textsc{Introduction to Data Science in Industry} \\ [25pt] % Your university, school and/or department name(s)
\horrule{0.5pt} \\[0.4cm] % Thin top horizontal rule
\huge Homework 2 - Linear Algebra and Probability Review\\ % The assignment title
\horrule{2pt} \\[0.5cm] % Thick bottom horizontal rule
}

\author{Dorian Goldman} % Your name

\date{\normalsize\today} % Today's date or a custom date

\begin{document}

\maketitle % Print the title

%----------------------------------------------------------------------------------------
%	PROBLEM 1
%----------------------------------------------------------------------------------------

\section{Problem 1 - Linear Algebra Review}

%\lipsum[2] % Dummy text





\section{Problem 2 - Write a Python script and push it to the Github Classroom Repo}

\begin{itemize}
\item If you haven't already, install Anaconda on your machine, which gives you iPython notebook: \url{https://www.continuum.io/downloads}
\item Create a file called test.py in the folder where you've created your new repo (for the assignment, not the course).
\item Make your script test.py return the product of two numbers, entered via the command line. \textbf{Eg: \$ python test.py 3 4 } should return \textbf{12}. 
\item Add this file to the repo python-introduction-\{your-github-username\} by typing \emph(make sure you are in the directory python-introduction-\{your-github-username\}) :
\item \$ git add test.py
\item \$ git commit -m 'Added my homework'
\item \$ git push origin master.
\end{itemize}

\section{Problem 3 - Create a notebook that reads in a csv file with pandas}

In this exercise, you will create an iPython notebook. You will need to have installed Anaconda. To begin, open up Anaconda and launch Jupyter notebook.

\begin{itemize}
\item Within the Jupyter notebook, import the pandas library and load in the csv here:\\ \url{https://archive.ics.uci.edu/ml/machine-learning-databases/adult/adult.data}
\item Using the groupby clause within pandas, find the percentage of people who are making above \$50k according to their \emph{gender, race and country}.
\item Push your notebook to the repo python-introduction-\{your-github-username\}  following the steps you took in Problem 2. 
\end{itemize}

\textbf{If you need help:} Here is a sample notebook you can use to get started: \url{https://github.com/doriang102/Columbia_Data_Science/blob/master/homework/Homework-1-help.ipynb}


\end{document}